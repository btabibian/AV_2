When presented a motion sequence, moving hand is first detected in each image of the sequence. This is done in a number of steps.

At first image background is subtracted from the image. This allows us to only see things which have changed comparing to the background. Example of image from which background was subtracted is shown in Figure \ref{fig:bgsub},

Obtained image is then smoothened using median blur in order to reduce noise. Such image is then transformed into black and white image according to threshold based on a colour of a typical hand in the data set. Such colour is found to lie between RGB values of $(0, 0, 27)$ and $(100, 100, 140)$.

To improve precision of the background subtraction and to account for possible lighting changes between different sets of images, first image of the sequence is also subtracted from the image. This removes all static features from the image. Such static features are often artifacts caused by changed lighting conditions. Both, first image which is subtracted from image of interest and image of interest itself, are preprocessed in the same way.

\begin{math}
Image^*_n = f(Image_n - Background) - f(Image_1 - Background), n \geq 2
\end{math}

