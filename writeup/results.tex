\graphicspath{{./results/}}
Figures \ref{fig:paperSeq}, Figure \ref{fig:rockSeq} and Figure \ref{fig:scissorsSeq} show different images sampled from each \textit{rock}, \textit{paper}, \textit{scissors} signs respectively. Each set includes an original image, background-subtracted image and thresholded image after morphology is performed on the data. The line on the subtracted image demonstrates the detected boundary of the hand.

From the images it can be seen that boundary does not correspond exactly with the true boundary of the hand. This is the effect of dilation using cross kernel. In case of \textit{scissors} sign dilation helps connect two fingers which often appear disjoint due to a shadow between them.

\begin{figure}[htp]
\begin{center}
    \subfloat[Original]{\includegraphics[width=0.25\textwidth]{13orig.jpg}}
    \subfloat[Subtracted]{\includegraphics[width=0.25\textwidth]{13sub.jpg}}
    \subfloat[Thresholded]{\includegraphics[width=0.25\textwidth]{13bw.jpg}}\\
    \subfloat[Original]{\includegraphics[width=0.25\textwidth]{18orig.jpg}}
    \subfloat[Subtracted]{\includegraphics[width=0.25\textwidth]{18sub.jpg}}
    \subfloat[Thresholded]{\includegraphics[width=0.25\textwidth]{18bw.jpg}}\\
    \subfloat[Original]{\includegraphics[width=0.25\textwidth]{115orig.jpg}}
    \subfloat[Subtracted]{\includegraphics[width=0.25\textwidth]{115sub.jpg}}
    \subfloat[Thresholded]{\includegraphics[width=0.25\textwidth]{115bw.jpg}}\\
    \subfloat[Motion History Image]{\includegraphics[width=0.25\textwidth]{1mhi.jpg}}
\end{center}
\caption{Images for paper action}
\label{fig:paperSeq}
\end{figure}

\begin{figure}[htp]
\begin{center}
    \subfloat[Original]{\includegraphics[width=0.25\textwidth]{43orig.jpg}}
    \subfloat[Subtracted]{\includegraphics[width=0.25\textwidth]{43sub.jpg}}
    \subfloat[Thresholded]{\includegraphics[width=0.25\textwidth]{43bw.jpg}}\\
    \subfloat[Original]{\includegraphics[width=0.25\textwidth]{64orig.jpg}}
    \subfloat[Subtracted]{\includegraphics[width=0.25\textwidth]{64sub.jpg}}
    \subfloat[Thresholded]{\includegraphics[width=0.25\textwidth]{64bw.jpg}}\\
    \subfloat[Original]{\includegraphics[width=0.25\textwidth]{69orig.jpg}}
    \subfloat[Subtracted]{\includegraphics[width=0.25\textwidth]{69sub.jpg}}
    \subfloat[Thresholded]{\includegraphics[width=0.25\textwidth]{69bw.jpg}}\\
    \subfloat[Motion History Image]{\includegraphics[width=0.25\textwidth]{6mhi.jpg}}
\end{center}
\caption{Images for rock action}
\label{fig:rockSeq}
\end{figure}

\begin{figure}[htp]
\begin{center}
    \subfloat[Original]{\includegraphics[width=0.25\textwidth]{82orig.jpg}}
    \subfloat[Subtracted]{\includegraphics[width=0.25\textwidth]{82sub.jpg}}
    \subfloat[Thresholded]{\includegraphics[width=0.25\textwidth]{82bw.jpg}}\\
    \subfloat[Original]{\includegraphics[width=0.25\textwidth]{89orig.jpg}}
    \subfloat[Subtracted]{\includegraphics[width=0.25\textwidth]{89sub.jpg}}
    \subfloat[Thresholded]{\includegraphics[width=0.25\textwidth]{89bw.jpg}}\\
    \subfloat[Original]{\includegraphics[width=0.25\textwidth]{817orig.jpg}}
    \subfloat[Subtracted]{\includegraphics[width=0.25\textwidth]{817sub.jpg}}
    \subfloat[Thresholded]{\includegraphics[width=0.25\textwidth]{817bw.jpg}}\\
    \subfloat[Motion History Image]{\includegraphics[width=0.25\textwidth]{8mhi.jpg}}
\end{center}
\caption{Images for scissors action}
\label{fig:scissorsSeq}
\end{figure}

\graphicspath{{./}}

\begin{figure}
\begin{center}
\includegraphics[width=110mm]{paclassplot.png}
\caption{Scatter plot shows dependency between first two principal components used and actual class of an object. Black plus signs indicate "paper" sign , green crosses indicate "rock" sign, red circles indicate "scissors" sign. }
\label{fig:paclassplot}
\end{center}
\end{figure}

Cross validation test results showed that best accuracy over validation data set was achieved by using combination of Hu moments and temporal area of the extracted object as features of the motion image and multi-class SVM for classification. Results for using different feature sets are shown in Table \ref{tab:features}, for using different number of principal components are shown in Table \ref{tab:pca} and for using different classifiers in Table \ref{tab:classify}.

\begin{table}
\begin{center}
\begin{tabular}{| l | r | r |}
\hline
Features & Test set & Validation set \\ \hline
Temporal area of object only & 73.9\% & 54.2\% \\
Hu moments only & 79.3\% & 58.3\% \\
Both & 91.8\% & 87.5\% \\
\hline
\end{tabular}
\end{center}
\caption{Cross validation results when using different sets of features with optimal number of principal components and optimal classifier.}
\label{tab:features}
\end{table}


\begin{table}
\begin{center}
\begin{tabular}{| l | r | r |}
\hline
Number of PC & Test set & Validation set \\ \hline
3 & 87.3\% & 66.7\% \\
4 & 91.8\% & 79.2\% \\
5 & 91.8\% & 87.5\% \\
6 & 95.7\% & 73.3\% \\
7 & 96.0\% & 83.3\% \\
\hline
\end{tabular}
\end{center}
\caption{Cross validation results when using different number of principal components with set of features and optimal classifier.}
\label{tab:pca}
\end{table}

\begin{table}
\begin{center}
\begin{tabular}{| l | r | r |}
\hline
Classifier & Test set & Validation set \\ \hline
Logistic regression & 88.6\% & 66.7\% \\
Multi-class SVM & 91.8\% & 87.5\% \\
\hline
\end{tabular}
\end{center}
\caption{Cross validation results when using different classifiers with optimal set of features and optimal number of principal components.}
\label{tab:classify}
\end{table}


\subsubsection*{Discussion}

Because of combination of two different methods at feature extraction level we were able to achieve high accuracy of classification. While individually using Hu moments and temporal area of object gives us only moderate 58.3\% and 54.2\% accuracy respectively, combining both feature extraction methods and then performing PCA on them allows us to classify motion images with very high accuracy of 87.5\%. 

We were able to fine tune other parameters, such as number of principal components used and what type of classifier to use, to further maximise accuracy of classification.

\subsubsection*{Possible improvements}

Only using linear classification solvers was attempted for this assignment. Alternative classification algorithms, such as MLP or RBF networks, could be explored. 

Furthermore, motion history image could be improved. At the moment, every frame of a video clip is weighted the same without regards to its temporal location. However, frames located in the middle of a video clip typically hold largest information content as they are displaying the hand actually showing one of the signs. This could be exploited by giving frames close to the middle of a video clip higher weights than frames close to the beginning or end of the clip.

